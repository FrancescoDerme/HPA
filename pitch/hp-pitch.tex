\documentclass{article}
\usepackage[english]{babel}
\usepackage[letterpaper,top=2cm,bottom=2cm,left=3cm,right=3cm,marginparwidth=1.75cm]{geometry}
\title{hp-adaptive pitch}
\author{Pietro Fumagalli, Francesco Derme}

\begin{document}
\maketitle
\noindent
Hello everyone, I'm Pietro Fumagalli and with my colleague Francesco Derme we'll present our plan to build a C++, FEM-based, hp-adaptive PDE solver for multi-physics simulations, so let's dive into it. First of all, what's hp-adaptivity? The idea is to harness the freedom granted by the discontinuous Galerkin FEM to modify h (the diameter of a mesh element) and p (the polynomial degree of the approximant on the element), and to do so independently among different elements.
\newline
\newline
Let's consider the simple case of a Laplace equation on a square domain. We construct a uniform mesh to begin with and then locally refine or relax it depending on the expected behavior of the solution, so take a really small h and a small polynomial degree p where the solution is more problematic and a big h coupled with a high polynomial degree where the solution is more benign.
\newline
\newline
Now let's go a step further: consider a time-dependent heat equation with the addition of a non-linear reaction term, we want to employ the same technique which means refining the mesh at the "wave fronts" and relaxing it elsewhere as shown by the video (you can see here the wave travelling and the refinement following it). Of course, this comes with a major new difficulty: we have to estimate where the wave fronts are at a given time t, but we don't have the exact solution, so we'll define an indicator, say I(x), that varies over the mesh elements and predicts the behavior of the solution through a series of heuristics and then use a clustering algorithm on the values of I(x) to divide the mesh elements into the group where the wave fronts are located and the group where they aren't, and this has to be done a-priori for each time step. After having predicted where the wave fronts are we need to refine the mesh in those particular elements, this poses a series of mathematically challenging question: what is the optimal way to split an element into smaller ones? How to choose the right h and p? How we can de-refine the mesh once the wave fronts have moved on?
\newline
\newline
Now take all of this and consider it in a 3-d setting: here is where we find our direct applications. We'll apply this technique to wave equations that model the shocks that take place in the brain during an epileptic seizure. The hope is to create a reliable and computationally efficient approach to solve a very difficult problem in a complicated domain. 

\end{document}
